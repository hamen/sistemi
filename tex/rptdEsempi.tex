\begin{figure}[!h]
  \centering
  \subfloat[]{\label{fig:gradoAbilitazioneInfinito}
      \begin{tikzpicture}[node distance=1.3cm,>=stealth',bend angle=45,auto]
        \tikzstyle{place}=[circle,thick,draw=blue!75,fill=blue!20,minimum size=6mm]
        \tikzstyle{red place}=[place,draw=red!75,fill=red!20]
        \tikzstyle{transition}=[rectangle,thick,draw=black!75,
          fill=black,minimum size=4mm]
        
        \tikzstyle{every label}=[red]
        \node [place,label=above:$P_1$] (p1)      {};
        
        \node [transition,label=above:$t_1$][label=below:$\vartheta
          \equiv 1.5$] (t1) [left of=p1] {}
        edge [post] (p1);
      \end{tikzpicture}
  }
  \subfloat[]{\label{fig:gradoAbilitazione2}
    \begin{tikzpicture}[node distance=1.3cm,>=stealth',bend angle=45,auto]
      \tikzstyle{place}=[circle,thick,draw=blue!75,fill=blue!20,minimum size=6mm]
      \tikzstyle{red place}=[place,draw=red!75,fill=red!20]
      \tikzstyle{transition}=[rectangle,thick,draw=black!75,
        fill=black,minimum size=4mm]
      
      \node [place,label=above:$P_1$] (p1) {};
      \node [place,label=above:$P_2$] (p2) [right of=p1,xshift=1.3cm]{};
      
      \node [transition,label=above:$t_1$][label=below:$\vartheta_1
        \equiv 2$] (t1) [right of=p1] {}
      edge [pre, bend right] (p1)
      edge [post, bend left] (p1)
      edge [post] (p2);
    \end{tikzpicture}
  }
  \caption{Reti di Petri temporizzate}
  \label{fig:retiPetriTemporizzate}
\end{figure}
