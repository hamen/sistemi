\begin{figure}[!h]
  \centering
  \begin{tikzpicture}[node distance=1.3cm,>=stealth',bend angle=45,auto]

  \tikzstyle{place}=[circle,thick,draw=blue!75,fill=blue!20,minimum size=6mm]
  \tikzstyle{red place}=[place,draw=red!75,fill=red!20]
  \tikzstyle{transition}=[rectangle,thick,draw=black!75,
  			  fill=black!20,minimum size=4mm]

  \tikzstyle{every label}=[red]
  
  \node [place,label=above:$P_1$] (p1) {};
  \node [place,label=above:$P_2$] (p2) [above of=p1,yshift=0.4cm]{};

  \node [transition,label=above:$t_1$] (t1) [above left of=p1] {}
  edge [pre] (p2)
  edge [post] (p1);
  
  \node [transition,label=above:$t_2$] (t2) [above right of=p1] {}
  edge [pre] (p1)
  edge [post] (p2);
  \end{tikzpicture}
  \caption{Contatore di Petri}
  \label{fig:magazzinoPetri}
\end{figure}
